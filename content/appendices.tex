% !TeX root = ../SDonchezThesis.tex

\appendix

\chapter{EDF Development Container Specification}\label{apx:EDFDocker}
\begin{lstlisting}[language=docker, breaklines=true, caption={Dockerfile for EDF Development Container}, label=lst:Dockerfile]
FROM ubuntu:18.04

# build with "docker build --build-arg PETA_VERSION=2020.2 --build-arg PETA_RUN_FILE=petalinux-v2020.2-final-installer.run -t petalinux:2020.2 ."

# install dependences:

ARG UBUNTU_MIRROR
RUN [ -z "${UBUNTU_MIRROR}" ] || sed -i.bak s/archive.ubuntu.com/${UBUNTU_MIRROR}/g /etc/apt/sources.list 

RUN apt-get update &&  DEBIAN_FRONTEND=noninteractive apt-get install -y -q \
  build-essential \
  sudo \
  tofrodos \
  iproute2 \
  gawk \
  net-tools \
  expect \
  libncurses5-dev \
  update-inetd \
  libssl-dev \
  flex \
  bison \
  libselinux1 \
  gnupg \
  wget \
  socat \
  gcc-multilib \
  libidn11 \
  libsdl1.2-dev \
  libglib2.0-dev \
  lib32z1-dev \
  libgtk2.0-0 \
  libtinfo5 \
  xxd \
  screen \
  pax \
  diffstat \
  xvfb \
  xterm \
  texinfo \
  gzip \
  unzip \
  cpio \
  chrpath \
  autoconf \
  lsb-release \
  libtool \
  libtool-bin \
  locales \
  kmod \
  git \
  rsync \
  bc \
  u-boot-tools \
  python \
  xinetd \
  tftpd \
  tftp \
  libsm6 \
  libxi6 \
  libxrandr2 \
  libfreetype6 \
  libfontconfig \
  libswt-gtk-4-java \
 && apt-get clean \
 && rm -rf /var/lib/apt/lists/*

RUN dpkg --add-architecture i386 &&  apt-get update &&  \
      DEBIAN_FRONTEND=noninteractive apt-get install -y -q \
      zlib1g:i386 \
    && apt-get clean \
    && rm -rf /var/lib/apt/lists/*

ARG PETA_RUN_FILE
ARG VIVADO_TAR_FILE
ARG VIVADO_VERSION

RUN locale-gen en_US.UTF-8 && update-locale

#make a Vivado user
RUN adduser --disabled-password --gecos '' vivado && \
  usermod -aG sudo vivado && \
  echo "vivado ALL=(ALL) NOPASSWD: ALL" >> /etc/sudoers

COPY .devcontainer/accept-eula.sh /tmp/
COPY .devcontainer/${VIVADO_TAR_FILE}.tar.gz /
COPY .devcontainer/install_config_vitis.txt /
COPY .devcontainer/install_config_petalinux.txt /

#config tftp server
COPY .devcontainer/tftp.in /etc/xinetd.d/tftp
RUN mkdir /tftpboot
RUN chmod ugo+rw /tftpboot/
RUN service xinetd stop
RUN service xinetd start

# run the install

RUN echo "Extracting Vivado tar file" 
RUN tar xzf ${VIVADO_TAR_FILE}.tar.gz 
RUN rm ${VIVADO_TAR_FILE}.tar.gz
RUN ${VIVADO_TAR_FILE}/xsetup --agree 3rdPartyEULA,WebTalkTerms,XilinxEULA --batch Install --config install_config_vitis.txt 
RUN ${VIVADO_TAR_FILE}/xsetup --agree 3rdPartyEULA,WebTalkTerms,XilinxEULA --batch Install --config install_config_petalinux.txt
RUN mkdir /tools/Xilinx/PetaLinux/${VIVADO_VERSION}/install/
RUN chmod -R 777 /tools/Xilinx/PetaLinux/${VIVADO_VERSION}/
RUN chmod -R 777 /tmp/
RUN cd /tmp && \
  sudo -u vivado -i /tmp/accept-eula.sh /tools/Xilinx/PetaLinux/${VIVADO_VERSION}/bin/${PETA_RUN_FILE} /tools/Xilinx/PetaLinux/${VIVADO_VERSION}/install/
RUN rm -f /tools/Xilinx/PetaLinux/${VIVADO_VERSION}/bin/${PETA_RUN_FILE} /tmp/accept-eula.sh
RUN rm -rf ${VIVADO_TAR_FILE}*

# make /bin/sh symlink to bash instead of dash:
RUN echo "dash dash/sh boolean false" | debconf-set-selections
RUN DEBIAN_FRONTEND=noninteractive dpkg-reconfigure dash

USER vivado
ENV HOME /home/vivado
ENV LANG en_US.UTF-8

#add vivado tools to path
RUN echo "service xinetd start" >> /home/vivado/.bashrc
RUN echo "source /tools/Xilinx/Vitis/2020.2/settings64.sh" >> /home/vivado/.bashrc
RUN echo "source /tools/Xilinx/PetaLinux/2020.2/install/settings.sh" >> /home/vivado/.bashrc

#setup vivado license
RUN mkdir /home/vivado/.Xilinx
COPY .devcontainer/Xilinx.lic /home/vivado/.Xilinx/
\end{lstlisting}

\chapter{AXI Masters Supported by the XMPU-PL IP Core}\label{apx:axi-masters}
The table below indicates the various AXI masters supported by the XMPU-PL IP core. These masters are supported in both the lock bypass (for XMPU-PL configuration) and AXI Master-based region restriction registers. This table is recreated from \cite{noauthor_memory_2021}.

\begin{table}[ht!]
  \centering\begin{tabular}{|c|c|c|c|l|}
      \hline
      Field Name & Bits & Type & Reset Value & \multicolumn{1}{c|}{Description} \\
      \hline
      Reserved & 31 & ro & 0x0 & Reserved \\
      MID\_FPD\_DMA[6:7] & 30 & rw & 0x0 & Enable FPD DMA [ch 6:7] \\
      MID\_FPD\_DMA[4:5] & 29 & rw & 0x0 & Enable FPD DMA [ch 4:5] \\
      MID\_FPD\_DMA[2:3] & 28 & rw & 0x0 & Enable FPD DMA [ch 2:3] \\
      MID\_FPD\_DMA[0:1] & 27 & rw & 0x0 & Enable FPD DMA [ch 0:1] \\
      MID\_DP\_DMA[4:5] & 26 & rw & 0x0 & Enable DisplayPort DMA [ch 4:5] \\
      MID\_DP\_DMA[2:3] & 25 & rw & 0x0 & Enable DisplayPort DMA [ch 2:3] \\
      MID\_DP\_DMA[0:1] & 24 & rw & 0x0 & Enable DisplayPort DMA [ch 0:1] \\
      MID\_PCIE & 23 & rw & 0x0 & Enable PCIe \\
      MID\_DAP\_AXI & 22 & rw & 0x0 & Enable Debug Access Port AXI \\
      MID\_GPU & 21 & rw & 0x0 & Enable GPU \\
      MID\_SATA1 & 20 & rw & 0x0 & Enable SATA1 \\
      MID\_SATA0 & 19 & rw & 0x0 & Enable SATA0 \\
      MID\_APU & 18 & rw & 0x0 & 
        \begin{tabular}{@{}l@{}}
          Enable APU \\ 
           \quad\textit{\textbf{Note}}: Requires that AxProt[1]=0. 
        \end{tabular}  
        \\
      MID\_GEM3 & 17 & rw & 0x0 & Enable GEM3 \\
      MID\_GEM2 & 16 & rw & 0x0 & Enable GEM2 \\
      MID\_GEM1 & 15 & rw & 0x0 & Enable GEM1 \\
      MID\_GEM0 & 14 & rw & 0x0 & Enable GEM0 \\
      MID\_QSPI & 13 & rw & 0x0 & Enable QSPI \\
      MID\_NAND & 12 & rw & 0x0 & Enable NAND \\
      MID\_SD1 & 11 & rw & 0x0 & Enable SD1 \\
      MID\_SD0 & 10 & rw & 0x0 & Enable SD0 \\
      MID\_LPD\_DMA[6:7] & 9 & rw & 0x0 & Enable LPD DMA [ch 6:7] \\
      MID\_LPD\_DMA[4:5] & 8 & rw & 0x0 & Enable LPD DMA [ch 4:5] \\
      MID\_LPD\_DMA[2:3] & 7 & rw & 0x0 & Enable LPD DMA [ch 2:3] \\
      MID\_LPD\_DMA[0:1] & 6 & rw & 0x0 & Enable LPD DMA [ch 0:1] \\
      MID\_DAP\_APB & 5 & rw & 0x0 & Enable Debug Access Port APB \\
      MID\_USB1 & 4 & rw & 0x0 & Enable USB1 \\
      MID\_USB0 & 3 & rw & 0x0 & Enable USB0 \\
      MID\_PMU & 2 & rw & 0x1 & Enable PMU \\
      MID\_RPU1 & 1 & rw & 0x0 & Enable RPU1 \\
      MID\_RPU0 & 0 & rw & 0x0 & Enable RPU0 \\
      \hline
  \end{tabular}
  \caption{Memory Ranges in the Isolation Design}
  \label{table:XMPU_PL_AXI_Masters}
\end{table}