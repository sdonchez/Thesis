% !TeX root = ../SDonchezThesis.tex

\chapter{Conclusion}\label{ch:conclusion}
It is apparent that cloud computing has revolutionized the way large scale computing is performed. One need only look at the massive scale of Amazon Web Services and Microsoft Azure to see that it is a computing paradigm that continues to see tremendous growth with no sign of slowing. Similarly, hardware acceleration and related applications of programmable logic in heterogenous systems alongside traditional processors is a field which has seen great interest in recent years. Therefore, it should come as no surprise that these two domains are colliding, with Cloud Service Providers beginning to offer FPGAs-as-a-service, either in a standalone capacity or integrated into a larger heterogenous system.

What is almost as immediately inherent is that the traditional model of FPGA-based development, with a single physical device per user, is at odds with the cloud community's transition away from single purpose hardware in favor of virtualization, in order to harness the economies of scale as well as the flexibility and scalability afforded therein. To this end, this research proposes an efficient multi-tenant architecture for FPGA-based cloud computing, realizing the benefits of both of these paradigms. By leveraging the power of virtual FPGAs, the architecture enables the CSP to provision tenant instances of varying size and capability, independent of the specific hardware available. Furthermore, it enables the CSP to utilize large form factor hardware, minimizing support overhead and increasing the scalability of the system.

The design proposed by this effort improves on those proposed in contemporary academic research in a number of ways. It acknowledges the necessity of tenant security by enabling bitstream encryption in conjunction with robust memory and peripheral isolation, without compromising the CSPs overhead. It does so by means of schedulable decryption assets in an effort to maximize available footprint and minimizing supervisory logic idle times. Furthermore, it offers enhanced security over its contemporary designs, further reducing the vulnerable surface by which a tenant's IP can be compromised by a malicious actor, whether that actor be a co-tenant, an outside threat, or the CSP itself.

\section{Avenues for Future Research}
Although this research effort outlines a comprehensive architecture for the implementation of an efficient multi-tenant FPGA-based cloud computing system, it by no means attempts to fully address each component of that architecture, as such an attempt would far exceed the volume of this work and the scale of the effort as a whole. Rather, it attempts to focus on key portions of the architecture that have either been to this point unaddressed, or which represent a divergence from the current larger body of academic research. Accordingly, several topics presented in the architecture (specifically enumerated in Section \ref{subsec:Obj}) serve as avenues for future research. 

Specifically, the integration of the actual AES and KAC cores controlled by the scheduler of Chapter \ref{ch:edfScheduling} and protected by the memory isolation logic outlined in \ref{ch:dmaProtection} remain an important avenue for future work. An implementation of the former is available from Xilinx (although licensing is required), however, to the best of this effort's knowledge, no HDL-based implementation of the latter is currently available in either industrial use or academic research.

Similarly, implementation of a secure enclave in the processing system, such as would be afforded by a Trusted Execution Environment (TEE), remains an avenue for such research. Although open-source TEE implementations are readily available, the integration of such a system into the environment already implemented by this effort remains an open research issue. Such an implementation would greatly enhance the integrity of the security-sensitive portions of the system, such as the allocation of the AES and KAC cores, and the protection of the memory isolation logic.

