% !TeX root = ../SDonchezThesis.tex

%Maggie to revise
\pagestyle{empty}
\chapter{Abstract}\label{ch:abstract}

Field Programmable Gate Arrays (FPGAs), and, more generally, Multi-Processor System on a Chip (MPSoC) devices (in which FPGAs play an integral role), have long been utilized for the implementation of hardware-accelerated functionality in embedded systems. As much of the global technology industry has transitioned to cloud-based infrastructure over the past decade, many cloud providers have realized the potential revenue source that FPGAs as a service could provide. Current cloud FPGA offerings mimic the traditional FPGA design paradigm, wherein a single physical device is allocated to a single user. This represents a fundamental departure from the traditional cloud computing architecture, wherein large, powerful compute resources are subdivided and allocated to multiple tenants. To date, there has been some attempt in the research community to develop a multi-tenant cloud based MPSoC solution that would satisfy these security concerns. 

% However, designs to date have required numerous decryption cores that consume valuable space in the FPGAs themselves, and which likely have low utilization rates on account of their distribution across the system. Furthermore, these works do not fully satisfy the security concerns of the tenant, as they rely on trust in the tenant-cloud Service Provider (CSP) relationship, where financial incentive may cast that relationship into doubt.

This work seeks to propose a new architecture for multi-tenant FPGA-based MPSoCs in a cloud setting. This architecture draws from those currently proposed in academia, but introduces fundamental changes to afford a more resource-efficient multi-tenant solution that simultaneously reduces overhead while also negating the security concerns outlined above. It does so through resource sharing and efficient scheduling, along with the use of a robust security framework that segregates tenants from each other while ensuring trust in the overarching system design.

These goals are specifically addressed in this work through in-depth exploration of several fundamental aspects of the architecture. The use of shared decryption engines allocated through a modified Earliest Deadline First (EDF) scheduling algorithm is presented, which is tailored to work in real-time on aperiodic tasks. This enables a tremendous reduction in resource utilization on the FPGA fabric, enabling scalability that was previously unrealistic. Additionally, memory isolation logic and a trusted static bitstream are proposed, which protect tenants from malicious cotenants, as well as shift the trusted relationship from the CSP to the FPGA Vendor, whose sole financial incentive is to ensure the integrity of the tenant data. Finally, a brief discussion of Key-Aggregation Cryptography presents an overview of a little-discussed cryptosystem that has the potential to enable highly efficient IP confidentiality for tenants, without substantial overhead on the part of any party.