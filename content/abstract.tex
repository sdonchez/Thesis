% !TeX root = ../SDonchezThesis.tex

\pagestyle{empty}
\chapter{Abstract}\label{ch:abstract}
Field Programmable Gate Arrays (FPGAs) have long been utilized for the implementation of hardware-accelerated functionality in embedded systems. They afford efficient prototyping due to their rapid reconfigurability, and enable tremendous cost savings as compared to custom hardware. As much of the global technology industry has transitioned to cloud-based infrastructure over the past decade, many cloud providers have realized the potential revenue source that FPGAs as a service could provide. Current cloud FPGA offerings mimic the traditional FPGA design paradigm, wherein a single physical device is allocated to a single user. This represents a fundamental departure from the traditional cloud computing architecture, wherein large, powerful compute resources are subdivided and allocated to multiple tenants. The use of single-user FPGAs in the cloud to date has chiefly been due to privacy concerns, as hardware-based design opens up whole new avenues for malicious actors to compromise a system.

To date, there has been some attempt in the research community to develop a multi-tenant cloud based FPGA solution that would satisfy these security concerns. However, designs to date have required numerous decryption cores that consume valuable space in the FPGAs themselves, and which likely have low utilization rates on account of their distribution across the system. Furthermore, these works do not fully satisfy the security concerns of the tenant, as they rely on trust in the tenant-cloud Service Provider (CSP) relationship, where financial incentive may cast that relationship into doubt.

This work seeks to extend the architectures currently proposed in academia to afford a more resource-efficient multi-tenant solution that simultaneously reduces overhead while also negating the security concerns outlined above. It does so through the use of shared decryption engines allocated through a modified Earliest Deadline First (EDF) scheduling algorithm, in conjunction with memory isolation logic and a trusted static bitstream in order to shift the trusted relationship from the CSP to the FPGA Vendor, whose sole financial incentive is to ensure the integrity of the tenant data.