% !TeX root = ../SDonchezThesis.tex

\chapter{Direct Memory Access (DMA) Protection}\label{ch:dmaProtection}
Although the EDF Algorithm outlined in the preceding chapter enables the efficient decryption of tenant bitstreams with shared decryption engines, the encrypted transmission of the bitstream is itself useless if any co-tenant on the PFPGA is granted access to the decrypted bitstream as stored in RAM. To this end, it is imperative that memory isolation functionality be utilized throughout the PFPGA to limit each tenant's partition to only the portions of the memory space that are allocated to said tenant. Similarly, care should be taken to re-allocate access permissions to the decryption engines as they change tasks, such that the end user can be guaranteed that access only persists through the decryption partition for as long as is necessary. Finally, the static partition, containing the CSP's supervisory functionality and resources, should be, to the maximum extent practical, isolated from the tenant data (although the presence of the HPS interface and the DDR controller interface in this region necessitate some amount of unrestricted access).

To this end, the architecture presented in Chapter \ref{ch:systemArchitecture} incorporates the use of Xilinx Memory and Peripheral Protection Units for Programmable Logic Isolation (XMPU-PL), as proposed in \cite{noauthor_memory_2021}. By combining these cores with individual AXI interconnect switches for each partition, and configuring the Region parameters of each core according to the partition it is located in, total isolation can be achieved provided that the static bitstream is from a known and trusted source (as has been outlined elsewhere in this work).

The remainder of this chapter outlines the implementation of this mechanism in the programmable logic. It begins with a discussion of the relevant background material and other academic work in the area. It then outlines in detail the proposed design, and concludes with a discussion of its implementation and the results of testing.

\section{Background}\label{sec:DMABackground}

\section{Proposed Memory Isolation Design}\label{sec:DMADesign}

\section{Development and Test Environment}\label{sec:DMAEnvironment}

\section{Implementation and Experimental Results}\label{sec:DMAResults}