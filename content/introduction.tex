% !TeX root = ../SDonchezThesis.tex

\chapter{Introduction}\label{ch:introduction}

Field Programmable Gate Arrays, or FPGAs, have long been utilized in devices benefiting from hardware acceleration of processes unsuitable for execution on a traditional processor, but which don't warrant the time consuming and cost-prohibitive process of ASIC design. Historically, such devices have been used extensively for prototyping,as well as in production devices, but generally in an environment where they are integrated into a dedicated device and intended for a single purpose. This closely parallels the history of much of traditional computing, wherein systems are geographically co-located with their users and associated with a single entity. 

However, the last decade has seen a massive shift in traditional computing away from on-premise compute resources in favor of third-party managed cloud computing. This transition has allowed increased flexibility and cost-efficiency for enterprise and personal users alike, and only continues to grow in ubiquity as time goes on. Accordingly, it should come as no surprise that, as much of the world pivots from on-site datacenters and computing resources to hybrid or cloud based platforms, Cloud Service Providers (CSPs) have shown an interest in providing FPGA-as-a-service resources to their tenants. Amazon, Google, and Microsoft all offer such services as part of their respective cloud platforms, as do many of the other major players in the industry. 

Currently, these offerings mirror the traditional FPGA implementation - a single physical device per tenant instance. This is due to a variety of factors, mostly centered around device security for both the tenant and the CSP. However, research is underway that seeks to utilize Dynamic Partial Reconfiguration (DPR) as a means to enable CSPs to partition large FPGAs into multiple virtual devices, each of which can be allocated to a tenant. This would allow CSPs to utilize much more economical devices, driving down cost for tenants and likely spurring increased adoption. Before this technology can see widespread adoption, it must address a number of security concerns surrounding co-tenancy, as well as trust relationships between the CSP and the tenant.

This effort seeks to address the latter concern, that of the CSP - tenant trust relationship. Understandably, many tenants are hesitant to provide their Intellectual Property (IP) directly to the CSP, as there is financial incentive for a CSP to integrate some of that IP into their own accelerator offerings. Furthermore, tenants may be limited by internal or governmental regulations with regards to the transmission of sensitive IP to other parties, including CSPs. 

Accordingly, tenants seek to upload encrypted IP to their partitions, with assurances that it is not possible for the CSP to directly access the decrypted content. This is feasible, and, even within the constraints of the multi-tenant scenario outlined above, has already been researched in academia. However, the solution posed in the literature is inefficient in its duplication of components, and furthermore contains a flaw that entirely violates the integrity of the bitstream as it pertains to its inaccessibility in its decrypted form by the CSP. This research effort seeks to address that vulnerability, while also providing performance enhancements to the original design. Specifically, this work, as a part of the larger effort, seeks to provide an efficient way of allocating shared decryption resources in a manner that reduces overhead while still ensuring the integrity of the IP's encryption through all CSP-accessible parts of the tenancy cycle.

\section{Organization of this Work}\label{sec:organization}
This work explores several individual facets of the architecture outlined above and expanded on in the subsequent chapters. Chapter \ref{ch:relatedWork} explores the state of academia with regards to the field of cloud-based multi-tenant FPGA devices, including a detailed analysis of the prior work which proposes the architecture this effort seeks to extend. Chapter \ref{ch:systemArchitecture} outlines the proposed enhancements to this architecture in detail, in preparation for the subsequent chapters which each discuss a particular aspect thereof.

Specifically, Chapter \ref{ch:edfScheduling} outlines the novel EDF scheduling algorithm proposed as a result of this research effort, which is utilized to allocate shared decryption resources between tenants equitably. Meanwhile, Chapter \ref{ch:dmaProtection} explores the use of memory isolation functionality as made available in recent FPGAs, which greatly enhances the capacity of the CSP to provide meaningful isolation of data between tenants. Chapter \ref{ch:keyAggregateCryptography} explores the use of a key-aggregate cryptosystem to facilitate the easy encryption and subsequent decryption of the tenant bitstreams without requiring large key storage facilities to be implemented by the CSP. Chapter \ref{ch:conclusion} concludes this work, and provides a brief discussion of avenues for future research in this area.